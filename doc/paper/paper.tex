% 
% Annual Cognitive Science Conference
% Sample LaTeX Paper -- Proceedings Format
% 

% Original : Ashwin Ram (ashwin@cc.gatech.edu)       04/01/1994
% Modified : Johanna Moore (jmoore@cs.pitt.edu)      03/17/1995
% Modified : David Noelle (noelle@ucsd.edu)          03/15/1996
% Modified : Pat Langley (langley@cs.stanford.edu)   01/26/1997
% Latex2e corrections by Ramin Charles Nakisa        01/28/1997 
% Modified : Tina Eliassi-Rad (eliassi@cs.wisc.edu)  01/31/1998
% Modified : Trisha Yannuzzi (trisha@ircs.upenn.edu) 12/28/1999 (in process)
% Modified : Mary Ellen Foster (M.E.Foster@ed.ac.uk) 12/11/2000
% Modified : Ken Forbus                              01/23/2004
% Modified : Eli M. Silk (esilk@pitt.edu)            05/24/2005
% Modified : Niels Taatgen (taatgen@cmu.edu)         10/24/2006
% Modified : David Noelle (dnoelle@ucmerced.edu)     11/19/2014

%% Change "letterpaper" in the following line to "a4paper" if you must.

\documentclass[10pt,letterpaper]{article}

\usepackage{cogsci}
\usepackage{pslatex}
\usepackage{apacite}


\title{TODO title}
 
\author{{\large \bf Jan Gosmann (jgosmann@uwaterloo.ca)} \\
  %Department of Psychology, 1202 W. Johnson Street \\
  %Madison, WI 53706 USA
  \AND{\large \bf Aaron Voelker (TODO)} \\
  \AND{\large \bf Chris Eliasmith (TODO)}
  %Department of Educational Psychology, 1025 W. Johnson Street \\
  %Madison, WI 53706 USA}
  }


\begin{document}

\maketitle


\begin{abstract}
The abstract should be one paragraph, indented 1/8~inch on both sides,
in 9~point font with single spacing. The heading ``{\bf Abstract}''
should be 10~point, bold, centered, with one line of space below
it. This one-paragraph abstract section is required only for standard
six page proceedings papers. Following the abstract should be a blank
line, followed by the header ``{\bf Keywords:}'' and a list of
descriptive keywords separated by semicolons, all in 9~point font, as
shown below.

\textbf{Keywords:} 
add your choice of indexing terms or keywords; kindly use a
semicolon; between each term
\end{abstract}


\section{Introduction}
Winner-take-all (WTA) mechanisms are often employed in cognitive models. For 
example, TODO O'Reilly, Nengo cleanup, TCM\@. A WTA mechanism receives 
a $d$-dimensional input of utility values for $d$ different choices. The output 
is supposed to be larger than zero for the dimension with highest utility and 
zero for all other inputs.

A large body of literature exists examining how well different WTA mechanisms 
can fit different data from decision experiments (TODO references). Here, 
however, we will investigate the suitability of two different WTA mechanisms in 
the context of large-scale cognitive modelling with spiking neurons on a set of 
benchmarks that are more normative in nature.  The first mechanism is an 
implementation of the classic Usher-McClelland (TODO ref) leaky, competing 
accumulator model.  It has been widely used, for example in versions of the TCM 
model (TODO ref), and TODO some of our models. We will compare this to an 
independent accumulator model with a second thresholding layer with recurrent 
connections to the first layer. In both cases we are especially interested in 
situations with more than two ($d > 2$) choices.

For the implementation of these models we use the Neural Engineering Framework 
(NEF) which allows us to directly implement the prescribed dynamics with spiking 
neurons. The NEF has been used in a wide range of models TODO, including the 
largest functional brain model Spaun (TODO ref). We will give a short 
introduction to the NEF first, then describe the implementation of the two WTA 
mechanisms. In section TODO, we give the results on a number of metrics, 
followed by a discussion in section TODO\@.

\section{Methods}

\subsection{The Neural Engineering Framework}

\subsection{Leaky, competing accumulator model}

\subsection{Independent accumulator model}


* ever increasing evidence

\section{Acknowledgments}

Place acknowledgments (including funding information) in a section at
the end of the paper.


\section{References Instructions}

Follow the APA Publication Manual for citation format, both within the
text and in the reference list, with the following exceptions: (a) do
not cite the page numbers of any book, including chapters in edited
volumes; (b) use the same format for unpublished references as for
published ones. Alphabetize references by the surnames of the authors,
with single author entries preceding multiple author entries. Order
references by the same authors by the year of publication, with the
earliest first.

Use a first level section heading, ``{\bf References}'', as shown
below. Use a hanging indent style, with the first line of the
reference flush against the left margin and subsequent lines indented
by 1/8~inch. Below are example references for a conference paper, book
chapter, journal article, dissertation, book, technical report, and
edited volume, respectively.

\nocite{ChalnickBillman1988a}
\nocite{Feigenbaum1963a}
\nocite{Hill1983a}
\nocite{OhlssonLangley1985a}
% \nocite{Lewis1978a}
\nocite{Matlock2001}
\nocite{NewellSimon1972a}
\nocite{ShragerLangley1990a}


\bibliographystyle{apacite}

\setlength{\bibleftmargin}{.125in}
\setlength{\bibindent}{-\bibleftmargin}

\bibliography{CogSci_Template}


\end{document}
